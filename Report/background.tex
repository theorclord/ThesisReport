\section{Background}
\subsection{Definitions}
It is commonly admitted that there are as many definitions of games as there are players. People working in the game industry, researchers, and players still disagree on what key elements should a taxonomy of games be based on. The french sociologist Roger Caillois (1958, p.3) explains the difficulty of establishing an all-inclusive game classification:
\begin{quotation}
The multitude and infinite variety of games at first causes one to despair of discovering a principle of classification capable of subsuming them under a small number of well-defined categories. Games also possess so many different characteristics that many approaches are possible.
\end{quotation} 
With this statement, one could argue that defining or classifying game genres is not relevant, as there will always be examples that do not fit the definition, which will then need to be extended and will therefore move away from its original meaning. This paper will certainly not bring a clear and universal definition of the game genres relevant to its topic, as definitions and taxonomies are necessarily affected by their research purpose. However, basic definitions are needed to build an understandable framework and provide tools that will help when analysing the impact of the use of procedural content generation in tabletop games, both on their design and creation and on the experience they enable.
Obviously, the implementation of digital components into tabletop games can already cause problems to some people, depending on their understanding on what tabletop games are. The following paragraphs will narrow down the definitions, in order to point out the key elements that will be referred to in the analysis of the contribution of PCG in Archipelago. Those key elements will be then used to point out existing games applicable to the topic of this paper.

\subsubsection{Tabletop Games}
The definition of "tabletop games" is the first one that must be refined in order to correctly comprehend the purpose of this paper. An example is needed as a reference in order to point out their key elements. Nat Levan, designer at Oakleaf Games gives his own definition of tabletop games". (Levan, 2014)
\begin{quotation}
The defining feature of a tabletop game is the need for components, not the need for a table. [...] so perhaps saying that a tabletop game is a game with a structured placement of components establishes a better requirement.
\end{quotation} 
"Tabletop games" is a generic term used to refer to a broad variety of game genres that are played on a table or the like. They range from card games, board games, not forgetting miniature games or dice games. The aim of defining "tabletop games" this way is also a way to exclude digital games. Although removing the need for a table from the definition seems rather counter-intuitive, designers make this assumption to include other simple games using components. Nat Levan uses the example of the game of "guess what I’m holding in my hand"(Levan, 2014). Here is a perfect example of a definition that spreads out of its original purpose in order to be inclusive. According to that definition, the game of marble for example could also be called a "tabletop game". Reducing a game to the use of its components is as unsatisfying as saying that a tabletop game is only made to be played on a table. For instance, digital adaptations of tabletop games are not a new thing. As an example, \textit{Magic: The Gathering} (MicroProse, 1997) is a digital version of the Wizards of the Coast’s trading card game of the same name. Moreover, websites allowing traditional tabletop games like \textit{Carcassone} (Klaus-Jürgen Wrede, 2000), \textit{Through the Ages: A Story of Civilization} (Vlaada Chvátil, 2006) or \textit{Race for the Galaxy} (Tom Lehmann, 2007) and others to be played online have made their appearance with the development of the internet. These games do not cease to be "tabletop games", even though the play experience is made different by changing the context in which they are played.


The next step of the process is to define what can be called "components", and why they have to be central in the game. Core components of tabletop games are generally thought as physical. They can be cards, tokens, pawns, or any object used as a central element in the game, which cannot be played without them.  As we know, tabletop games have generally evolved staying away from digital components. Exceptions exist though, and game makers have already experimented the use of technology to influence their design. The game \textit{Atmosfear} (Brett Clements, 1991) included a VHS tape showing "The Gatekeeper", who attributes rewards to the players, punishes them or gives them instructions at specific moments in the game, thus also acting as a timer. In that sense the VHS can be seen as a component in itself, like a deck of cards with a hourglass. 
Components could then be defined as interactive elements without which the game cannot be played. The central dimension of components is clearly understandable, but it is important to notice that their physical property has been excluded from the definition.


After this reflection, it is now possible to come up with a definition of what tabletop games are. Tabletop games cannot be narrowed to the simple use of physical components, neither must they only be played on a table; a more appropriate definition could be that tabletop games are games requiring the players to use central components in a space - real or virtual - representing a table or similar surface.


\subsubsection{Board Games}

"Board games" are generally considered as a subcategory of tabletop games. The Oxford Dictionaries (2016) provide the following definition:
\begin{quotation}
"A game that involves the movement of counters or other objects round a board."
\end{quotation}
While this is a rather broad definition, it can still be used to present key aspects of board games. The terms "Counters or other objects" refer to the core components that were previously mentioned as a key aspect of tabletop games. The new element refining the definition being the game board, which is the surface on which the players move pieces, place cards or tiles in order to achieve the goal of the game. As it has been explained beforehand, this surface is not necessarily physical and can be then represented on a digital device like a screen. In a board game, the board is of course a key component. It is the center of the attention of players, and can serve several purposes. 
The board concentrates all the structural elements that the players should focus on in order to correctly play the game. It must support the play experience as much as put limitations on the way the game is played. It that sense it is as much an interface from which players get the more or less abstract information they need, as it is an obstacle forcing them to play according to the rules. In the game of Chess for example, those properties are clearly identifiable, though the board is not the most detailed. It limits the movement of the pieces within a square of 8 units of length, and a trained eye can easily identify valuable information (i.e. movement possibilities or strategy planning) , thanks to the tiled structure of the board. One last property that can be identified, is that a board is rarely a core component in itself. During a game, the board only make sense when it is accompanied by other core components like tokens or cards. Boards have different structures and purposes, and listing their properties here would not be relevant. But from those identified earlier, it becomes possible to give a definition of "board games": "board games" are tabletop games using a game board and its properties as a core component.

\subsubsection{Collaborative Games}
Though their approach on the play experience is somewhat similar, researchers traditionally make the difference between cooperative and collaborative games. Nash (2002, cited in Zagal et al., 2006, p.25) argues that in cooperative games,the players have interests that are "neither completely opposed, nor completely coincident". In contrast,"in a collaborative game, all the participants work together as a team, sharing the pay-offs and outcomes; if a player wins or loses, everyone wins or loses"(Zagal and Rick,2006, p.25). This differentiation between "cooperation" and "collaboration" makes a big difference in the play experience. As an example, \textit{Diplomacy} (Allan B. Calhamer, 1959) encourages players to create alliances between them, and betray them when it is needed. However, in \textit{Forbidden desert} (Matt Leacock 2013), players have to manage the resources they individually possess, and put their skills at the service of the team. 
The differentiation is usually made in the game's winning condition. Cooperative games are still based on competition, whereas collaborative games have only winners or only losers. Collaboration involves more open discussions among the players and exclude the necessity to voluntarily hide information. 


\subsection{Hybrid Board Games}
"Hybrid games" is an expression that has many different definitions. Therefore, there is not a universal understanding of what hybrid games are. Some people simply call "hybrid games" games that are a combination, or hybridization, of two or more different genres of games. The interest of that definition is that it removes the analogue or digital aspect of games and simply focuses on the experience they provide. It needs to be refined though, because many games can be seen as a combination of genres, as game designers often build upon already existing game mechanics, trying to adapt them in another context. In this paper, "hybrid board games" will be referred as board games (and all their properties previously defined) using one or several digital components. One criteria needs to be added though: as it has been already mentioned, the digital or analogue nature of the components or the board should not be what defines a hybrid game. A game is much more than its components, gameplay mechanics or even its rules. A game can be defined by the experience that the combinations of all these factors enable. Therefore, only board games including digital components in order to enable an experience that cannot exist without using the computing power of digital devices will be considered as "hybrid board games" in this paper. Games that are only using digital devices to simulate analogue game components (deck of cards or board for example) without providing any new feature are not considered hybrid games. Several games fitting this definition have already been created. The three examples presented later show the potential of this approach of hybrid games, each in its own way.
\subsubsection{False Prophets (Mandrik and Maranan, 2000}
People have also started to refer to "hybrid games" as games that try to close the gap between analogue and digital games. False Prophets is an experimental game that explores "the space between board games and video games, leveraging (sic)the advantages of both"(2000, Mandrik and Maranan). This experiment was a technical challenge more than a game design experiment. Mandryk and Maranan created a tabletop display using sensors in order to recreate an interface. A map is projected on a surface containing infra-red sensors, which can detect the position of the tokens in the displayed tiles. Originally, the map is hidden. It is only when the game's computer detects players approaching its borders that the map expands. The interest of this experiment is that it tries to exploit the assets of both game boards and computers. It exploit the setting of having an interface around which players can gather, and computation can be used to do complex calculation that would be very complex to do for humans. In this case, the manifestation of that is the map that gets revealed dynamically. While fully analogue games  that possess this feature, like \textit{Zombies!!!}(2001, Todd Breitenstein and Kerry Breitenstein) already exist, the sensor interface could allow many possibilities for game mechanics. The creators of False Prophets mentioned having "playing pieces that store data, interpret contextual information, and travel continually with the players"(Mandrik and Maranan, 2000).
\subsubsection{STARS (Magenkurth Stenzel and Prante, 2003)}
STARS is another example of the integration of digital devices in the traditional way of playing tabletop games. However, this project is an attempt to create an "augmented tabletop game platform", which could be usable with multiple games. The platform features an interactive touch-sensitive screen, which displays the content of the game board as well as it manages the interactive playing pieces. A camera is situated over the table and tracks the position of other playing pieces. STARS also features a wall display and other digital components like a small Personal Digital Assistant (used by the players to secretly communicate and store personal information related to the game), and audio devices. The experiment is an attempt to exploit a ubiquitous computing environment (an environment exploiting multiple devices) to allow the creation of more complex game rules, and recreate a game world with more fidelity without slowing down the game flow with multiple complex calculation. It also allows multiple game sessions, as well as game persistence. Although this experiment looks similar to False Prophets, its interest is that it exploits multiple devices and their affordances in order to create a platform which can then be used for multiple games. It opens to the possibility of creating other tabletop games using digital devices as core components and enable new experiences. One could argue that the quantity of available information (on the board, the wall display and the personal digital assistant) might get in the way of play by making it too complex. But this invention is interesting in its approach of hybrid board games: it uses the potential of digital devices as core components, enabling new play experiences that would hardly be possible otherwise.
\subsection{Procedural Content Generation}
"Procedural content generation (PCG) refers to the algorithmic creation of content. It allows content to be generated automatically, and can therefore greatly reduce the increasing workload of artists." Rolan van der Linden et al, Procedural generation of dungeons.
