\section{Background}
\subsection{Definitions}
Common definitions of board games; Theory related to the way they are commonly played...
Other (board) games using pcg; Specially if there are some board games that have apps. 
Theory: Alexander's design patterns concept.
Related work with PCG.
Papers about game content generation
Theory: PCG for designers
Theory: PCG based Game Design
Algorithms used.
References to papers ( L-system, other pcg methods used in games)
Board games of similar manner
Prototyping and Data collection 
Theory: Houde and Hill on prototypes
Theory: Anatomy of prototypes
Theory: Anatomy of a failure (playtests)
Observation (Qualitative)
It is commonly admitted that there are as many definitions of games as there are players and game designers. People in the game industry, researchers, and players still disagree on what key elements should a taxonomy of games be based on. In The Defintion of Play and The Classification of Games (), the french sociologist Roger Caillois explains the difficulty of establishing an all-inclusive game classification as follows:

“The multitude and infinite variety of games at first causes one to despair of discovering a principle of classification capable of subsuming them under a small number of well-defined categories. Games also possess so many different characteristics that many approaches are possible.”

With this statement, one could argue that defining or classifying game genres is not relevant, as there will always be examples that do not fit the definition, which will then need to be extended and will therefore move away from its original meaning. This paper will certainly not bring a clear and universal definition of the game genres relevant to its topic, as definitions and taxonomies are necessarily affected by their research purpose. However, basic definitions are needed to build an understandable framework and provide tools that will help when analysing the impact of the use of procedural content generation in tabletop games, both on their design and creation and on the experience they deliver.
Obviously, the implementation of digital components into tabletop games can already cause problems to some people, depending on their understanding on what tabletop games are. The following paragraphs will narrow down the definitions, in order to point out the key elements that will be referred to in the analysis of the contribution of PCG in [game title]. Those key elements will be then used to point out existing games applicable to the topic of this paper.

\subsubsection{Tabletop games}

“The defining feature of a tabletop game is the need for components, not the need for a table. [...] so perhaps saying that a tabletop game is a game with a structured placement of components establishes a better requirement.” Nat Levan, game designer at OakLeaf games.

“Tabletop games” is a generic term used to refer to a broad variety of game genres that are played on a table or the like. They range from card games, board games and not forgetting miniature games or dice games. The goal of defining “tabletop games” this way is also a way to exclude digital games. Although removing the need for a table from the definition seems rather counter-intuitive, designers make this assumption to include other simple games using components. Nat Levan uses the examples of the game of “guess what I’m holding in my hand”. Here is a perfect example of a definition that spreads out of its original purpose in order to be inclusive. According to that definition, the game of marble could also be called a “tabletop game”. Reducing a game to the use of its components is as unsatisfying as saying that a tabletop game is only made to be played on a table. For instance, digital adaptations of tabletop games are not a new thing. As an example, Magic: The Gathering (1997, MicroProse) is a digital version of the Wizards of the Coast’s trading card game of the same name. Moreover, websites allowing traditional tabletop games like Carcassone (2000, Klaus-Jürgen Wrede), Through the Ages: A Story of Civilization (2006, Vlaada Chvátil) or Race for the Galaxy (2007, Tom Lehmann) and others to be played online have made their appearance with the development of the internet. 

The next step of the process is to define what can be called “components”, and why they have to be central in the game. Tabletop games’ core components are generally seen as physical. They can be cards, tokens, pawns, or any object used as a central element in the game, which cannot be played without them.  As we know, tabletop games have generally evolved staying away from digital components. Exceptions exist though, and game makers have already experimented the use of technology to influence their design. The game Atmosfear (Brett Clements, 1991) included a VHS tape showing “The Gatekeeper”, who attributes rewards to the players, punishes them or gives them instructions at specific moments in the game, thus also acting as a timer. However, the game does not cease to be a tabletop game, as it is played on a flat surface and uses components. In that sense the VHS can be seen as a component in itself, like a deck of cards with a hourglass. 
Components can then be defined as interactible elements without which the game cannot be played. The central dimension of components is clearly understandable, but it is important to notice that their physicality has been excluded from the definition.

After this reflexion, it is now possible to define what tabletop games are. Tabletop games cannot be narrowed to the simple use of physical components, neither must they only be played on a table; a more appropriate definition could be that tabletop games are games requiring the players to use central components in a space - real or virtual - representing a table or similar surface.

\subsubsection{Board Games}

“Board games” are generally considered as a subcategory of tabletop games. The Oxford Dictionaries provide the following definition:

“A game that involves the movement of counters or other objects round a board.”

While this is a rather broad definition, it can still be used to present key aspects of board games. The terms “Counters or other objects” refer to the central components that were previously mentioned as a key aspect of tabletop games. The new element refining the definition being the game board, which is the surface on which the players move pieces, place cards or tiles in order to achieve the goal of the game. As it has been explained beforehand, this surface is not necessarily physical and can be then represented on a digital device like a screen. In a board game, the board is of course a key component. It is the center of the attention of players, and can serve several purposes. 
The board concentrates all the structural elements that the players should focus on in order to correctly play the game. It must support the play experience as much as put limitations on the way the game is played. It that sense it is as much an interface from which players get the more or less abstract information they need, as it is an obstacle forcing them to play according to the rules. In the game of Chess for example, those properties are clearly identifiable, though the board is not the most detailed. It limits the movement of the pieces within a square of 8 units of length, and a trained eye can easily identify valuable information (i.e. movement possibilities or strategy planning) , thanks to the tiled structure of the board. One last property that can be identified, is that a board is rarely a core component in itself. During a game, the board only make sense when it is accompanied by other core components like tokens or cards. Boards have different structures and purposes, and listing their properties here would not be relevant. But from those identified earlier, it becomes possible to give a definition of “board games”: “board games” are tabletop games using a game board and its properties as a core component.


\subsubsection{Collaborative board games}

Though their approach on the play experience is somewhat similar, researchers traditionally make the difference between cooperative and collaborative games. In Collaborative games: lessons learned from board games, Zagal and Rick define a cooperative game as a game that “models a situation where two or more players have interests that are “neither completely opposed, nor completely coincident (Nash, 2002)””. In contrast, they state that “in a collaborative game, all the participants work together as a team, sharing the payoffs and outcomes; if a player wins or loses, everyone wins or loses”. This differentiation between “cooperation” and “collaboration” makes a big difference in the play experience. As an example, Diplomacy (1959, Allan B. Calhamer) encourages players to create alliances between them, and betray them when it is needed. However, in Forbidden desert (2013), players have to manage the resources they individually possess, and put their skills at the service of the team. 
The differentiation is usually made in the game’s winning condition. Cooperative games have generally one winner, whereas collaborative games have only winners or only losers. Collaboration involves more open discussions among the players and exclude hidden information. Cooperative games are still competitive and collaborative games are all about competing against the game itself. 


\subsection{Hybrid Board Games}

The expression “hybrid games” seems to have many different meanings. 




\subsection{Procedural Content Generation}

"Procedural content generation (PCG) refers to the algorithmic creation of content. It allows content to be generated automatically, and can therefore greatly reduce the increasing workload of artists." Rolan van der Linden et al, Procedural generation of dungeons.
