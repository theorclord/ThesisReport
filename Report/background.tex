\section{Background}
Common definitions of board games; Theory related to the way they are commonly played...
Other (board) games using pcg; Specially if there are some board games that have apps. 
Theory: Alexander's design patterns concept.
Related work with PCG.
Papers about game content generation
Theory: PCG for designers
Theory: PCG based Game Design
Algorithms used.
References to papers ( L-system, other pcg methods used in games)
Board games of similar manner
Prototyping and Data collection 
Theory: Houde and Hill on prototypes
Theory: Anatomy of prototypes
Theory: Anatomy of a failure (playtests)
Observation (Qualitative)
A :

It is commonly admitted that there are as many definitions of games as there are players and game designers. People in the game industry, researchers, and players still disagree on what key elements should the taxonomy of games be based on. In The Defintion of Play and The Classification of Games (), the french sociologist Roger Caillois explains the difficulty of establishing an all-inclusive game classification as follows:

"The multitude and infinite variety of games at first causes one to despair of discovering a principle of classification capable of subsuming them under a small number of well-defined categories. Games also possess so many different characteristics that many approaches are possible."

With this statement one could argue that defining or classifying game genres is not important, as there will always be examples that do not fit the definition, which will then need to be extended and will therefore move away from its original meaning. This paper will certainly not bring a clear and universal definition of the game genres relevant to its topic, as definitions and taxonomies are necessarily affected by their research purpose. However, basic definitions are needed to build an understandable framework and provide tools that will help when analysing the impact of the use of procedural content generation in tabletop games, both on their design and creation and on the experience they deliver.
Obviously, the implementation of digital components into tabletop games can already cause problems to some people, depending on their understanding on what tabletop games are. The following paragraphs will narrow down the definitions, in order to point out the key elements that will be referred to in the analysis of the contribution of PCG in [game title]. Those key elements will be then used to point out existing games applicable to the topic of this paper.

Tabletop games

"The defining feature of a tabletop game is the need for components, not the need for a table. [...] so perhaps saying that a tabletop game is a game with a structured placement of components establishes a better requirement." Nat Levan, game designers at OakLeaf games.

Tabletop games is a generic term used to refer to a broad variety of game genres that are played on a table or the like. They range from card games, board games and not forgetting miniature games or dice games. The goal of defining "tabletop games" this way is mostly a way to exclude video games from the reflection, as playing on a digital device . Although excluding the need for a table seems rather counter-intuitive, designers make this assumption to include other simple games using components. Nat Levan uses the examples of the game of "guess what I'm holding in my hand". Here is a perfect example of an definition that spreads out of its original purpose in order to be inclusive. According to that definition, the game of marble could also be called a "tabletop game". 

Reducing a game to the use of its components is as unsatisfying as saying that a tabletop game is only made to be played on a table. Adaptation of tabletop games are not a new thing. As an example, Magic: The Gathering (1997, MicroProse) is a digital version of the Wizards of the Coast's trading card game of the same name. Moreover, websites allowing traditional tabletop games like Carcassone (2000, Klaus-Jürgen Wrede), Through the Ages: A Story of Civilization (2006, Vlaada Chvátil) or Race for the Galaxy (2007, Tom Lehmann) and others to be played online have made their appearance with the development of the internet. 

Tabletop games cannot be narrowed to the simple use of components, neither must they only be played on a table. A more appropriate definition could be that tabletop games are games requiring the players to use central components representation in a space, real or virtual, representing a table or any similar surface.

Board Games

"Board games" are generally considered as a subcategory of tabletop games. The Oxford Dictionaries provide the following definition:

"A game that involves the movement of counters or other objects round a board."

While this is a rather broad definition, it can still be used to present key aspects of board games. "Counters or other objects" refers to the central components that were previously mentioned as a key aspect of tabletop games. The new element refining the definition being the board, which is the surface on which the game is played. As it has been explained beforehand, this surface is not necessarily physical and can be then represented on a digital device like a screen. 




\subsection{Hybrid Games}

The next step of the process is to define what can be called "components", and why they have to be central. Tabletop games' components are generally seen as physical. They can be cards, tokens, pawns, or any object used as a central element in the game, which cannot be played without them.  As we know, tabletop games have generally evolved staying away from digital components. Exceptions exist though, and game makers have already experimented the use of technology to influence their design. The game Atmosfear (Brett Clements, 1991) included a VHS tape showing "The Gatekeeper", who attributes rewards to the players, punishes them or gives them instructions at specific moments in the game, thus also acting as a timer. However, the game does not cease to be a tabletop game, as it is played on a flat surface and uses components. In that sense the VHS can be seen as a component in itself, like a deck of cards with a hourglass. 
More recently, the success of Hearthstone: Heroes of Warcraft (2014, Blizzard) has also shown the potential of using digital devices to influence the design of trading cards games. The animations, visual effects and sounds enhances the experience of playing traditional trading card games and allows a different way of providing feedback. The digital aspect of the game also contributes to the game in other ways. It allows the players to destroy the cards they do not need in order to create new ones.
