\section{Introduction}
Introduction to the ideas behind the project  ( Using PCG in a hybrid game)
Our problem statement ( Uncover a potential, adapt existing mechanics to a different media, explore a design space, create new mechanics / playful interactions…)
Why would you use PCG in a board game.
Most obvious challenges ahead
Uncovering less obvious problematics/challenges (thanks to the project)
Project scope and data gathering (Qualitative Methods)
The Roles of the group members.
Breaking the conventional way of board games
Thesis report layout. Short walkthrough of the report


PCG needs to be written with out the abbreviation in this section
An introduction to the name of our game



Some researchers try to think about analogue and digital games as a whole, and prefer talking about play as an activity instead of making a clear distinction between the two. This theory is crafted so that game studies are not distorted by the tendency of people to think about games in terms of mechanics or support (Stenros and Waern, 2010). In reality however, digital and analogue games are generally thought as two isolated categories of games.