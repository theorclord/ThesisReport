\section{Introduction}
Introduction to the ideas behind the project  ( Using PCG in a hybrid game)
Our problem statement ( Uncover a potential, adapt existing mechanics to a different media, explore a design space, create new mechanics / playful interactions…)
Why would you use PCG in a board game.
Most obvious challenges ahead
Uncovering less obvious problematics/challenges (thanks to the project)
Project scope and data gathering (Qualitative Methods)
The Roles of the group members.
Breaking the conventional way of board games
Thesis report layout. Short walkthrough of the report


PCG needs to be written with out the abbreviation in this section
An introduction to the name of our game



Some researchers try to think about analogue and digital games as a whole, and prefer talking about play as an activity instead of making a clear distinction between the two. This theory is crafted so that game studies are not distorted by the tendency of people to think about games in terms of mechanics or support (Stenros and Waern, 2010). In reality however, digital and analogue games are generally thought as two isolated categories of games.

At the moment this paper was written, a few games falling under the definition of hybrid games were available on the market. \textit{XCOM: The Board Game} (Lang, 2015)\cite{game:xcomtbg} and \textit{Golem Arcana} can be cited as examples of games using a smartphone to offer more variety in gameplay mechanics. Since the development of such digital devices is growing more and more rapidly, it was to be expected that board game makers would use this as a way to renew the tabletop game play experience in order to keep up with the evolving video game industry. Woods (2012, p.5) made a statement that can explain this growing interest for enhancing traditional board game experience with the use of more evolved component:

\begin{quotation}
For many people, board games are an anachronism. The idea of gathering around a table to move pieces on a board conjures up visions of a bygone era [...]. For some people, however, board games are far from being redundant. To them they are a hobby, a passion, even an obsession.
\end{quotation}

In general, people seem to refer to hybrid games as "digitally or electronically enhanced board games". 


So what does these digital enhancements offer? One of the most obvious answers to this, is memory and computation power. By having access to a digital device within a board game, you can store information on a detailed level that would otherwise be hard, if not impossible, if just using a piece of paper and a pen. It also opens up the possibility of generating content on the go as the games progress. By using procedural content generation (PCG) in a board game, the game can have a higher variety of contents available, without physically having to contain an excessive amount of pieces, tokens and cards. Procedural content generation opens up a world of possibilities. It can be used to generate stories, events, rules, outcomes and technically anything one might want to have. If you imagine a game of \cite{game:cah} and the hundreds of cards the physical box contains, along with all its expansion packs, and then imagine having to sort and physically handle all of them. The cards are many, indeed, and their variety is comprehensive, but if you were to take away all the white cards that the players handle, and instead replace them with their phone, what could that yield? It would allow the players to have a wider range of cards, without having to actually physically handle the cards. New cards could be procedurally generated, and they could even be based on past events and cards that the players frequently use. 
Of course, this might not be the best game to create a hybrid from, as the game requires a player to select a card without knowing who placed it down, a task that would be broken if every player use their own phones and all the phones are different. But if the game instead came with a number of identical looking hardware that contained the card-generating software, it is not so hard to imagine that the game could have a lot more diversity without having to consume physical space and time for ordering of all the cards.