\chapter{Introduction}
Introduction to the ideas behind the project  ( Using PCG in a hybrid game)
Our problem statement ( Uncover a potential, adapt existing mechanics to a different media, explore a design space, create new mechanics / playful interactions…)
Why would you use PCG in a board game.
Most obvious challenges ahead
Uncovering less obvious problematics/challenges (thanks to the project)
Project scope and data gathering (Qualitative Methods)
The Roles of the group members.
Breaking the conventional way of board games
Thesis report layout. Short walkthrough of the report

An introduction to the name of our game

Some researchers try to think about analogue and digital games as a whole, and prefer talking about play as an activity instead of making a clear distinction between the two. This theory is crafted so that game studies are not distorted by the tendency of people to think about games in terms of mechanics or support%\ref{ref to bibtex} 
(Stenros and Waern, 2010). In reality however, digital and analogue games are generally thought as two isolated categories of games.

At the time when this paper was written, a few games falling under the definition of hybrid games were available on the market. \textit{XCOM: The Board Game} (Lang, 2015)\cite{game:xcomtbg} and \textit{Golem Arcana} can be cited as examples of games using a smartphone to offer more variety in gameplay mechanics. Since the development of such digital devices is growing more and more rapidly, it was to be expected that board game makers would use this as a way to renew the tabletop game play experience in order to keep up with the evolving video game industry. Woods (2012, p.5) made a statement that can explain this growing interest for enhancing traditional board game experience with the use of more evolved component:

\begin{quotation}
For many people, board games are an anachronism. The idea of gathering around a table to move pieces on a board conjures up visions of a bygone era [...]. For some people, however, board games are far from being redundant. To them they are a hobby, a passion, even an obsession.
\end{quotation}

In general, people seem to refer to hybrid games as "digitally or electronically enhanced board games". 


So what does these digital enhancements offer? One of the most obvious answers to this, is memory and computation power. By having access to a digital device within a board game, you can store information on a detailed level that would otherwise be hard, if not impossible, if just using a piece of paper and a pen. 
It also opens up the possibility of generating content on the go as the games progress. By using procedural content generation (PCG) in a board game, the game can have a higher variety of contents available, without physically having to contain an excessive amount of pieces, tokens and cards. 
Procedural content generation opens up a world of possibilities. It can be used to generate stories, events, rules, outcomes and technically anything one might want to have. If you imagine a game of \textit{Cards Against Humanity}\cite{game:cah} and the hundreds of cards the physical box contains, along with all its expansion packs, and then imagine having to sort and physically handle all of them. The cards are many, indeed, and their variety is comprehensive, but if you were to take away all the white cards that the players handle, and instead replace them with their phone, what could that yield? It would allow the players to have a wider range of cards, without having to actually physically handle the cards. New cards could be procedurally generated, and they could even be based on past events and cards that the players frequently use. 

Of course, this might not be the best game to create a hybrid from, as the game requires a player to select a card without knowing who placed it down, a task that would be broken if every player use their own phones and all the phones are different. But if the game instead came with a number of identical looking hardware that contained the card-generating software, it is not so hard to imagine that the game could have a lot more diversity without having to consume physical space and time for ordering of all the cards.
While the hybrid game genre is developing, few games seem to have experimented the implementation of PCG to design core mechanics. 
\textit{Golem Arcana} only uses static content based on the chosen scenario and the miniatures used in the game.
\textit{XCOM: The Board Game} seems to generate content based on the entries feeding the application at the end of a phase, but it is hard to say how much content is generated thanks to the inputs. 

Another example that could be cited is the game \textit{Alchemists} (Kotry, 2014)\cite{game:alch} is a game in which the players have to identify which of the nine alchemical elements compose the ingredient used to create magic potions. The players must mix different ingredients in order to create functional potions, and push a theories that makes them earn reputation points. The player with the highest reputation after six rounds wins the game.

The details of the game play will not be explained here; instead it is necessary to focus on the implementation of PCG in order to illustrate the potential of such mechanics. Once a potion has been created by a player, it must be scanned with the camera integrated to the smartphone or tablet. If the ingredient \textbf{(CONTINUATION)}\\\\
\begin{description}
\item[Problem statement:]
Board games have limited space of play in accordance to the fact they are based on more or less static content. Our idea is to overcome this by exploring the creation of a hybrid board game  with app integration, using Procedural Content Generation (PCG) to create a dynamic and everchanging experience.

In the classical sense, board games has been about being analog. What we hope to do with this thesis, is to incorporate the digital into the analog world of board games. By doing so, we hope to close the gap between board games and digital games, and maybe open up the world to the possibility of mixing the digital with the analog. There has been some cases where the same concept has been explored before, i.e. X-com and Golem Arcana.

We want to create a board game concurrent with the creation of an app which is designed to be coherent with the boardgame. By using PCG we are hoping that the board game will feel as though it adapts itself according to the players' actions, giving it a sense of unpredictability and potentially infinite replay value. In that way, the application will not be a classical "board game supporting" application, but rather the central element of said board game.

Some of the challenges that lie before us, is to find relevant information about already existing solutions and implementations which deals with subjects which are similar to our project.
Another challenge in itself is to come up with the concept of a new board game that will be made from scratch. It must be engaging to the players, they should want to explore the game, and they should feel that they want to come back for more.

\end{description}


So who are we? This project group consists of three members; Maxime Moze, Mats Stenhaug and Mikkel Stolborg. All three are, at the time of writing this paper, master students at the IT University of Copenhagen, class of 2016. Maxime Moze will be the game designer of the group, and his main roles will be to create the board game with the physical pieces and layout, come up with the rules, mechanics and flow of the game, along with prototypes and input to the programmers on how the connection between the app and board should be. He will also be the one in charge of giving feedback to the programmers as to what functionality should be in the app, and user friendliness of the app.
The programmers in this group are Mikkel Stolborg and Mats Stenhaug. They will be in charge of coming up with the procedural algorithms for generating the content along with programming the application itself, and making sure the app is capable of utilizing memory in order to generate more content to be presented to the players. Mainly the roles of the programmers will be to create the logic behind the procedural content generation, and utilizing the algorithms in ways that will benefit the game as a whole, and give the players the best possible experience of using the digital aspect of the game, i.e. the application.
Together we will be communicating back and forth and work close together to ensure that the development process will go as smoothly as possible, and that there will be as little communication errors as possible.


In this paper, we will go through the entirety of the development process of the game \textit{Archipelago} with supporting theory about board games, hybrid games, and procedural content generation in general, and procedural content generation in \textit{Archipelago}. First the background section with related methodology will be presented, where in we talk about different game types as described above, definition of hybrid games, various types of PCG, a little summary of the game engine used to produce the application for the game, and game design methodology.
After that, we will go through all the details of how the game was produced. From brainstorming and design, the development process of the game, both the physical board game and the technical part that is the application. We will elaborate on the various PCG methods used to create the result that we came up with, including how the application is structured on a technical level. There will also be a step-by-step description of a play through of the game, and the various prototypes and play tests generated during the course of production.
Later in this paper there will be a discussion section, where we go through the various design choices of the game making process, we will elaborate on what is special about this game, how the different phases of it works, along with possible future work that could be done if this project were to be continued outside of the scope which is this thesis. You will be introduced to the PCG contribution and the benefits it hold, along with having memory in the app. Team dynamics and issues will also be discussed in this section.
In the end of this paper we will conclude the project by explaining how the PCG involvement contributed to the game, and we will relate the work done to the initial questions posed here.