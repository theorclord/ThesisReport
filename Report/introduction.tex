\chapter{Introduction}
\begin{quotation}
For many people, board games are an anachronism. The idea of gathering around a table to move pieces on a board conjures up visions of a bygone era [...]. For some people, however, board games are far from being redundant. To them they are a hobby, a passion, even an obsession. (Woods, 2012, p.5) \cite{book:euro}
\end{quotation}

Researchers try to think about analogue and digital games as a whole and prefer talking about play as an activity instead of making a clear distinction between the two. This theory is crafted so that game studies are not distorted by the tendency of people to think about games in terms of mechanics or support, but rather on the experience they enable (Stenros and Waern, 2010, p.1) \cite{art:stenroswaern}. In reality, however, digital and analogue games generally are conceptualized as two isolated categories of games, whereas they are fundamentally based on the same activity: play. 

Nonetheless, tabletop game makers have recently shown an interest in the advantages offered by widely spread digital devices such as smartphones and tablets, first by creating adaptations of existing board games. Even more recently, these game makers have started to exploit the possibilities offered by portable digital devices in the core of their design. This introduces the concept of hybrid games as it is understood in this paper. These games have started to draw the interest of a new type of audience.

One of the important things to notice is that there is not yet a clear definition of what hybrid games exactly are. Therefore, clearly defining the concept will be an important part of the research purpose. This definition process must help understanding why hybrid games deserve to be studied as a new genre of games allowing to create a new type of enjoyable play experience. 

The first problem that can be addressed is whether the digital enhancements can bring something new to the tabletop genre. One of the most obvious answers to this is memory and computation power. By having access to a digital device within a board game, information can be stored on a detailed level that would otherwise be hard to reach, if not impossible, if just using a piece of paper and a pen. The algorithmic concept called Procedural Content Generation (PCG) could well show the interest of exploiting the computing power of digital devices. 

PCG opens up a world of possibilities that can be exploited by hybrid games. It can be used to generate stories, events, rules, outcomes and others. Tabletop games could benefit from those possibilities and create variations of gameplay that could enable enjoyable experiences. The question that beckons is how a hybrid game could benefit from the content generation in the least intrusive way without the application feeling redundant. By using PCG in a board game, the game can have a higher variety of contents available, without physically having to contain an excessive amount of pieces, tokens, and cards. 

The purpose of this paper is to explore the challenges behind the creation of hybrid games in a direct and efficient way: by creating a hybrid game from the beginning and analyse the results of that creation. The creation of \textit{Archipelago} is the base to perform the experiment that this paper relies on. To support the gameplay and justify the use of a digital component, the game must also use PCG in a comprehensive way. Below is a constructed problem statement which is the foundation for the project, and outlines the set goal for the paper as a whole.

\begin{description}
\item[Problem statement:]

By creating a board game concurrent with the creation of an application which is designed to be coherent with said board game, we hope to explore the design space made available by exploiting PCG to enhance traditional tabletop game mechanics. In that way, the application must not be a classical \textit{board game supporting} application, but rather the central element of the board game. Following the same principle, this paper will also explore the technical challenges that will be uncovered throughout the game's creation process. 

One of the challenges is to find relevant information about already existing solutions and implementations which deal with subjects which are similar to this project. The information must include relevant theory related to board games, as well as already existing principles of game design based on PCG. Creating this framework must be the base of the future creation of \textit{Archipelago}. Another challenge in itself is to come up with the concept of a new board game that is made from nothing else than inspirations. There are not many existing solutions of hybrid games based on PCG, so \textit{Archipelago} must uncover the problems and propose a solution in its own framework. Finally, the discussion around the creation process should be a base on which to expand on to think about the creation of other hybrid games. 
\end{description}

In this paper, the development process of the game \textit{Archipelago} will be analysed, supported by theory about board games (and tabletop games in general), hybrid games, and procedural content generation. First, the background section with related methodology will be presented, wherein the relevant game genres are described in order to land on a common understanding of hybrid games. This background will also include the necessary theory about PCG and game design. These elements compose the framework of this paper.

After that, the development process of \textit{Archipelago} will be described. The aspects of its design process that are relevant to the purpose of the paper will be analysed. A technical overview of the creation of the application will also be provided. The various PCG methods used to create the resulting game need to be explained, as well as how the application is structured on a technical level. 

A crucial section of this paper will be the description of the final prototype that was built, based on the framework and the thought process of \textit{Archipelago}'s design and technical development. The final playtest using this prototype will be analysed, and combining the feedback and observations with other data collected while making the game will provide the necessary data to analyse the challenges of a PCG based hybrid game's creation.

A later chapter will go through various design choices relevant for the analysis of the game making process. The contribution of PCG in the special case of \textit{Archipelago} will be discussed, along with possible future work that could be done if this project were to be continued outside of the scope which is this thesis. Finally, and since the development of the game demands a particular combination of skills, a reflection on the team dynamics and development issues that are related to a hybrid game's development will be discussed.