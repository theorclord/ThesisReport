\chapter{Introduction}
Researchers try to think about analogue and digital games as a whole, and prefer talking about play as an activity instead of making a clear distinction between the two. This theory is crafted so that game studies are not distorted by the tendency of people to think about games in terms of mechanics or support (Stenros and Waern, 2010) \cite{art:stenroswaern}. In reality however, digital and analogue games are conceptualized as two isolated categories of games.

At the time when this paper was written, a few games falling under the definition of hybrid games were available on the market. Since the development of such digital devices is growing more and more rapidly, it was to be expected that board game makers would use this as a way to renew the tabletop game play experience in order to keep up with the evolving video game industry. Woods (2012, p.5) \cite{book:euro} made a statement that can explain this growing interest for enhancing traditional board game experiences with the use of digital components:

\begin{quotation}
For many people, board games are an anachronism. The idea of gathering around a table to move pieces on a board conjures up visions of a bygone era [...]. For some people, however, board games are far from being redundant. To them they are a hobby, a passion, even an obsession.
\end{quotation}

This statement shows that tabletop game play is an activity that people are still passionate with, and the activity itself is not something that should be left aside. But to bring around the same table the people having those two different visions of tabletop game play, game makers have first started to create digital adaptation of existing tabletop games. Later by using smartphones to create a new type of component, they have created what people started to refer to as \textit{digitally enhanced board games}. The first problem that can be addressed is whether the digital enhancements can bring something new to the tabletop genre. One of the most obvious answers to this, is memory and computation power. By having access to a digital device within a board game, information can be stored on a detailed level that would otherwise be hard, if not impossible, if just using a piece of paper and a pen. 
It also opens up the possibility of generating content on the go as the games progress. 

Procedural content generation (PCG) opens up a world of possibilities. It can be used to generate stories, events, rules, outcomes and others. Tabletop games could benefit from those possibilities and create variations of gameplay that could enable enjoyable experiences. The question that beckons is how a hybrid game could benefit from the content generation in the least intrusive way without the application feeling redundant. By using PCG in a board game, the game can have a higher variety of contents available, without physically having to contain an excessive amount of pieces, tokens and cards. 

The purpose of this paper is to explore the creation of hybrid games in the most drastic but efficient way: by creating a hybrid game from the beginning and analyse the results of that creation. To support the gameplay and justify the use of a digital component, the game must use PCG in a comprehensive way to support its mechanics. Below is a constructed problem statement which was the foundation for the project and outlines the set goal for the paper as a whole.

\begin{description}
\item[Problem statement:]

The idea behind this project is to explore the creation of a hybrid board game integrating a digital application and using Procedural Content Generation to create a dynamic and ever-changing experience. The purpose is to create a board game concurrent with the creation of an application which is designed to be coherent with the said board game. Using PCG must make the players feel that the game adapts itself according to the players' actions, giving it a sense of unpredictability and potentially infinite replay value. In that way, the application will not be a classical \textit{board game supporting} application, but rather the central element of said board game. The ultimate goal behind this project is to explore the design possibilities offered by such games, as well as identifying how to technically adapt PCG so that it fits with a game using both analogue and digital components.

One of the challenges is to find relevant information about already existing solutions and implementations which deals with subjects which are similar to this project. The information must include relevant theory related to board games, as well as already existing principles of game design based on PCG. Creating this framework must be the base of the future creation of \textit{Archipelago}. Another challenge in itself is to come up with the concept of a new board game that is made from nothing. There are not many existing solution of hybrid games based on PCG, so \textit{Archipelago} must uncover the problems and propose a solution in its own framework. Finally, this work should be a base on which to expand on to think about the creation of other hybrid games. 
\end{description}

%\So who are we? This project group consists of three members; Maxime Moze, Mats Stenhaug and Mikkel Stolborg. All three are, at the time of writing this paper, master students at the IT University of Copenhagen, class of 2016. Maxime Moze will be the game designer of the group, and his main roles will be to create the board game with the physical pieces and layout, come up with the rules, mechanics and flow of the game, along with prototypes and input to the programmers on how the connection between the app and board should be. He will also be the one in charge of giving feedback to the programmers as to what functionality should be in the app, and user friendliness of the app.The programmers in this group are Mikkel Stolborg and Mats Stenhaug. They will be in charge of coming up with the procedural algorithms for generating the content along with programming the application itself, and making sure the app is capable of utilizing memory in order to generate more content to be presented to the players. Mainly the roles of the programmers will be to create the logic behind the procedural content generation, and utilizing the algorithms in ways that will benefit the game as a whole, and give the players the best possible experience of using the digital aspect of the game, i.e. the application.Together we will be communicating back and forth and work close together to ensure that the development process will go as smoothly as possible, and that there will be as little communication errors as possible.

In this paper, the development process of the game \textit{Archipelago} will be analysed, supported by theory about board games (and tabletop games in general), hybrid games, and procedural content generation. First the background section with related methodology will be presented, where in the relevant game genres are described to land on a common understanding of hybrid games. This background will also include relevant theory about PCG and game design. These elements compose the framework of this paper.

After that, the development process of \textit{Archipelago} will be described. The aspects of its design process that are relevant to the purpose of the paper will be analysed. A technical overview of the creation of the application will also be provided. The various PCG methods used to create the resulting game need to be explained, as well as how the application is structured on a technical level. 

A crucial section ofthis paper will be the description is the final prototype that was built, based on the framework and the thought process of \textit{Archipelago}'s design and technical development. The final playtest using this prototype will be analysed, and combining the feedback and observations with other data collected while making the game will provide the necessary to analyse the challenges of a PCG based hybrid game's creation.

A later chapter will go through various design choices relevant for the analysis of the game making process. The contribution of PCG in the special case of \textit{Archipelago} will be discussed, along with possible future work that could be done if this project were to be continued outside of the scope which is this thesis. Finally, and since the development of the game demands a particular combination of skills, a reflection on the team dynamics and development issues that are related to a hybrid game's development will also be discussed in this section.