\section{Methods}
The progress and process of the project
\subsection{Brainstorming and design choices}
Co-op game 
Encourage discussion and exploit the tabletop setting (players sitting around a table)
2 phases game (management phase / exploration phase / resource input)
How the app could be used
\subsection{The Game - One play session}
From start to end. Full run of a game.
\subsection{Game design}
Extract/Specify the design patterns (of the 2 genres we are trying to "combine")
Designing how the app would work
Setting
Management system
Resources
Player Specialization
Risk/Reward
Infinite number of cards (events)
Event "memory"
Special events
UX (App/Board)
\subsection{App structure (How the game is composed (Technical))}
Scenes and their purpose

The scenes included in the app are as follows:
- StartScene
- WorldScene
- NodeScene
- EndScene

All of the scenes mentioned above will be explained and gone through in more details in this part. \\
\subsubsection{StartScene:}

As the game is started, you immediately enter the start screen. This is a simple scene containing only the button that says "Start Game". The purpose behind this scene is to let the users be able to chose when they want to start the game, so that they are not necessarily overpowered by the rules of the game (board game part) and the visuals of the app itself when started.
As soon as the "Start Game" button is clicked, the app will generate the contents of the world, and present it to the user within the next scene, which is the WorldScene. \\
\subsubsection{WorldScene:}

This is the "main" layout of the world. It contains all the locations that the players can travel, and shows them where they are, where they have been, and where they have to go. When you first see this part of the program, you will notice that it has floating islands with circles around it, as well as a flying castle. The castle is the piece the players control, and it is what the board is representing in the physical part of the game.
At first glance, you might not see all of the map and all the different islands you can travel to, but by simply zooming out, the players are able to get a better view of the entire map layout, and from there, they can plan their chosen way. There is also a little arrow in the lower right corner of the screen, which always will point towards the goal-island. 

Centered around the castle that the players control, you will find a large white circle. This circle is indicating the range that the castle can travel in one turn of the game. Any islands that are within said circle, are available to be travelled to.

When travelling from one island to another, the circle around the island you arrived at will turn green (from the initial white color). This signifies to the players that they have already been to that island and therefore do not necessarily have to go back to that place again unless they so choose. 
When you click an island on the map, you will be faced with a pop-up box. This box is containing the name of the island, and a piece of text that somewhat describes it; flavor text. The box then has 2 buttons, or choices, either "exit" or "travel". If you exit, the box is closed, and you are free to choose another island to travel to. If you click travel, however, your castle will then be moved towards that chosen island and stop once it reaches it. 
Once the castle has reached the island and the colliders on the object are triggered, the app will move on to the NodeScene, where the island that you travelled to will be represented. \\
\subsubsection{NodeScene:}

The NodeScene is probably best known as the "island scene" or "event scene", because 
this is the scene where all the events take place, and it is the main place where the players will have to interact with each other in order to choose their wanted event. 
To briefly describe the looks of the Scene, it has a background representation of an island, and on that island, it has image representations of the locations that the players will be able to travel to. On top of that, every Island (from the WorldScene) has a faction affiliated with it. What this does, is to allow the player to instinctively know whether or not an event will have a positive outcome and if it has a higher chance of success, based on the current standing the players have with that faction.
Once an area of the island has been clicked, let's say they chose a mine, a confirm box will be shown where the player will have to confirm that this is the area they want to explore. ((ABOUT THE EVENT TYPES HERE?))
After the confirmation, a new box will appear, where the event flavor text is presented along with either 2 or 3 options. Each option contains between 1 to 3 board pieces that will be utilized in order to complete the event. I.e. send 1 castle crew along with 1 alchemy point for a potion, etc. As soon as an option is selected, the event is concluded and the players are presented with the results; Success, Failure or Neutral. Each of these outcome types has rewards or penalties associated with them.
The last box that will appear in this scene is the result box where the players then have to click "return to world", and are then taken back to the WorldScene. \\
\subsubsection{EndScene:}

Upon reaching the final island in the WorldScene, the one island with the blue circle around it and that the arrow, that was mentioned earlier, is pointing towards, the game ends. What happens then is that all the data is cleared from the app, the WorldScene shifts to the EndScene, and the players are presented with an end text that basically tells them that they reached the goal. This scene also includes a button that takes the players back to the StartScene where they can play another game if they so choose.



\subsection{UI elements}

In this section, we will go through all the different UI elements within the app and explain in more detail their purpose and how they work.

\subsubsection{Boxes}

There are a number of different boxes that will appear over the course of a game session with this app. By "boxes" we refer to the dialog boxes that pop up when interacting with different elements of the application itself, e.g. an event popping up when interacting with an island.
The purpose of each box on a high level, is to convey information to the players, be it information about the location that they are travelling to, or the results of an event they have just completed.
Each box has it's own purpose, but as just stated, the main purpose is to convey information. The first box that appears when interacting with the app, is the "travel confirmation" box. This box tells a little information about the island that the player wishes to travel to, like the name of the island, and a few lines of flavor text that goes with said name. It also contains two buttons; one for accepting the travel location, and another one cancelling.
Once you have travelled to a location the next box will appear when the region the players want to explore is clicked. This box, like the one before, has a name (the location type like "mine") and a text field for the flavor about that mine. Again there is a button for accepting, and one for declining. 
Next up, you have the event box. This is the main box of the game, one might say. It contains a text field where the procedurally generated flavor text is displayed, along with 2 or 3 buttons that has the options for completion that the players may choose from. 
Once an option has been chosen, the application loads the result box. This box is mainly a field where the description of what happened during the event is shown. it also has a field where the results in terms of board pieces gained/lost, reputation changes, etc. are shown. In addition, it has a "return" button, where once clicked, the players are taken back to the WorldScene where their game continues.


\subsubsection{Faction icons}

The faction icons have a rather small visual part in the application, but the impact it makes on understanding the game and seeing what opportunities and rewards the players might get, is rather substantial. In the top left corner of each NodeScene screen, there is an icon that conveys to the players, which of the current 3 factions controls the island that they are on. With this information, and the information that the players might have a specific standing with said faction, allows the players to mentally make an image of what the outcomes of the event might be, in terms of good, bad or neutral.

\subsubsection{Buttons}	

The buttons in this application is the main way for the users to interact. If it is not the ray-casting and clicking on the islands in the world scene, the only other physical interaction with the application is through the buttons. The beauty of the buttons is that it allows the game to be rather easily built on to other platforms, such as android or IOS. The button has two main functionalities; the first would be to navigate within the app itself, like accepting to travel to one place, or closing one of the dialog boxes as described earlier. The other one is that the buttons themselves have a text field on them. By taking use of this feature, we are able to give the buttons meaning, in terms of carefully choosing what to display to the users. In the event box, there are between 2 or 3 buttons that holds options for how to solve the given event, and that means that we can set the text on the buttons to tell the players what each option will cost or require, e.g. one option can be to send out 1 of your crew to explore, whilst another option can be to send out 2 crew plus a cleric. Using precise and to-the-point text on the buttons allows the players to quickly understand what is required, and will allow for discussion to take place as to which one is the best option for the current state of the game.

\subsubsection{Islands}
	
The islands are the centre of the World Scene. All of them contains individually unique events that differ from each other. Each island is also controlled by a faction and as the players might get different standing with different factions at any given time in the course of a game, each tailored event will have a range of different outcomes depending on the players' standing with the controlling faction. 
As the main goal of the game, within the application, is to reach the goal node, it is important that the islands are spread around the world in a way so that the players themselves can choose which route they want to take in order to reach it. Be it the shortest way, or a longer way that requires more exploring and will take a longer time. Each island also have their own name and flavor to them, so that it hopefully will feel to the players that they have a history and a meaning.

\subsubsection{Castle}
	
This is the center piece on the table. The board in itself is a layout of the castle as represented within the application.
When moving around in the generated world, you move your entire castle. The castle is basically acting as the player object within the game. Since this is a collaborative game, however, it is up to the players to discuss amongst themselves how to use and distribute the resources that are "contained" within the castle, and use their own minds to select how an event is to be resolved when the castle is entering an island.

\subsubsection{Rings} 

Every island on the map has a ring around it. The color of the ring indicates whether the island is explored, unexplored or the goal, with the colors green, white and blue respectively. Having these rings is a way to signify to the users where they have been, where they can go, and gives the players a quick and easy overview of the map situation.
The castle also has a ring around it, but this is a significantly larger ring. This is because this ring signifies how far the castle is able to travel in one turn. Every island that is within that ring around the castle, is an island that they players are able to explore, and the map is made so, by the use of an L-System, that each node is possible to be visited, it is just up to the players to decide how and where they want to travel.

\subsubsection{Arrow} 

The functionality of the arrow located in the lower right corner of the world scene, is to give an easy way for the players to see which direction they need to go in order to reach the goal island.
This arrow is a nice little feature, because if the map is too big, and the users of the application were to focus the camera too far away from the map section, they will have an easy way to find the direction again. There are also other "safety" features implemented in case of camera issues, and they are described in the "Camera" section.

\subsubsection{Token Indicators (events)}
	
Functionality-wise, the token indicators are only there to allow the players to quickly get an idea of what physical board pieces are being used in any given event, whether it be in the conditions, or in the results. The tokens creates a link between the application and the physical part of the game by having recognizable images of the tokens that are identical to the ones actually used in the physical part. This again allows the players to get a notion of what is being used just by having a quick glimpse at the icons.

\subsection{Camera}

In this section, we will explain the uses of the camera and the functionalities that is implemented with it. 

\subsubsection{Centring on castle}

A problem that could occur when playing the game, was that the camera would go too far off to one side of the map. This could happen if a player would be reckless when moving the camera around the map in order to get a proper view of the situation. As a result of this problem being able to happen, there came a need to be able to quickly get back to the castle, so the solution was simple: Center the camera on the castle on command. So by simply double clicking anywhere on the map, the camera will reset it's position to the castle's current location. On top of having this, there is also the arrow, as described in the "UI Elements" section, pointing towards the end goal, so that if the players know the castle's relative position to the goal, they can navigate back to the castle if needed. 

\subsubsection{Zoom}
	
In order for the players to be able to get a good overview of the entirety of the map, there had to be a way for the camera to zoom out. However, zooming too far out would make the islands and the castle look too small, and not really give the proper resolution of the elements, so the maximum and minimum zoom distance are set within the app itself.	

\subsubsection{Reset Position}

(REMOVE, as same as the Center on castle?)
	
\subsubsection{Move around}

The players using the app are able to move the camera around all of the map in which ever direction they choose. This is also to encourage discussion and collaboration as to choosing which way they want to go, and which islands they want to explore; fastest way to the goal, or move around and explore first?
There is not set any max distance the camera can be from the castle, as the size of the map layout may vary depending on how the islands are organized and which layout is generated. This is also why the safety features of castle centring and the direction arrow are implemented.

\subsection{Datamanager}

The data manager is a collection of all the data saved locally within the application. In this section, we will elaborate on the purpose for it, how it is being accessed, and the overall usage of the manager and the data it contains.

\subsubsection{Purpose}
	
The purpose of having a data manager, is to have all data we want, in one convenient location. And by doing so, allowing all aspects of the application to have access to it. By allowing access from anywhere, all the data that is saved will be shared and available to use from wherever it is needed whenever it is needed. This also means that we can save data in real-time as for example an event is concluded.

\subsubsection{Usage}

There are several usages of the data manager, and they range from saving the islands, events, outcomes and selections, factions and standing to much more.

(Saving all the nodes, events, outcomes, selections, factions and their standing, castle and camera position, etc.)

\subsubsection{Access}
	
Accessing via Datamanager.instance: 	
	


World Nodes and Sub Nodes
Map Layout, spacing and placement
Raycasting?
\subsection{PCG Generating the content}
\subsubsection{L-system}
General Description of an L-system (Theory)

LRules

RuleCollection

States

Interpreting

Expanding
\subsubsection{Events}
The applications main purpose is to generate events which is the main interaction for the physical part of the game. The events provides resources for players and challenges in form of lost crew or constructed rooms.

Throughout the development of the game there have been several approaches to the problem of generating these events in an interesting manner and ensure the perceived space of possibilities was large enough for the events not to repeat themselves to often.

The first implementation used the location of the selected node to determine the events.
The idea was that the program was supplied with the location type, gathering, research, diplomacy, and from there a set of events were picked. 
The number of event options were randomly determined between 1 and 3. 
The event options generated came in three categories, basic, second, and third. This distinction was made to ensure the players always had the opportunity to chose at least one option. 
The options have a different list of outcomes tied to then, as to give some randomized outcome from selecting a button.
The outcomes, or results, were given a percent chance of appearing. 


The different approaches
Generating from type and selecting predefined events
Building from building blocks
Final version

The xml files.
How they are used. The scalability 
The different files
Flavor vs content

The structure of event generation
getting the type of event.
getting the conditions
getting the outcomes
saving the event


Combining the flavor text
Conditions and Outcomes (How they are connected)
Problem solving
Paper prototypes of building blocks(Strings)
Reverse engineering of text
Finding an acceptable way to combine
Saving data to produce new data
Having the right values
What is needed
How to access
How to build from it
\subsection{The app - From start to end}
The life of the app. 
Start game 
Generate nodes
Generate events
How an event is built
Moving around
Interacting with the islands
Flavor text combinations and it's purpose
Using memory to generate special event occurrences
