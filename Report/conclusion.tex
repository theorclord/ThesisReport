\chapter{Conclusion}
This is the final chapter of this paper, and in here we will conclude the project behind creating Archipelago. We will also conclude on how the project relates to the initial questions posed, and give some insight to what possible future work can be done.


\section{The exploration of hybrid games}
Introducing PCG to a hybrid game created a much bigger perceived possibility space within the contents of the game. Compared to regular hybrid games, where there is no specific PCG involved, \textit{Archipelago's} algorithms and structure allows for more content to be created while at the same time allows for the possibility for there to be pre-made content. 

In XCOM: The Board Game \cite{game:xcomtbg}, there are a number of pre-made events that can happen, and although there are some procedural aspects to how they are selected, they do not qualify as pure PCG the same way that Archipelago does. The difference here is that in XCOM: The Board Game, the event's occurrences are dependent on the feedback the application gets from the players. If certain criteria are not fulfilled, the game can not end. In Archipelago, the game ends when the players reach the goal island, which can be seen as criteria. However, since that \textit{criterion} is fulfilled by the players interacting with the application in the way that they do and the fact that there is an entire phase of the game which is entirely dependant on the application interaction, Archipelago is different. The procedurally generated content in Archipelago does not require input from the players in terms of telling the application whether or not certain criteria was completed, or if a base was destroyed. The application simply uses the selections that the players choose, and generates the content regardless of whether or not the players' castle is out of resources.

When developing both a physical and a digital game concurrently, one of the most important factors is that the mechanics design is thought through and that both the board game and the application have equal, yet distinctly specific purposes. As in \textit{Archipelago}, the board game had to rely on the application to handle the event generation while the application equally had to rely on the board game to utilize the values argued for within the logical data. 

In hybrid games, the two parts must be supporting each other, while at the same time standing out in their own respective areas with their afforded features being a key motivator when designing the game. 


\section{Exploration of PCG in hybrid games}
The possibilities that PCG provide during a project like this are immense. But building the program and discovering all the wanted features was a great task. In \textit{Archipelago} PCG was the biggest source of giving the players a setting in which they would immerse themselves. Since the events were the main connection between the board game and the application, it was only natural that this was the area that would bring the game to a whole. Having the events being generated procedurally made it possible to create a setup for future expansion of the game in terms of easily being able to add more small parts of content into the system. By having this fragmented structure, we were able to make the game system in a way that would allow for different themes and stories to be built. The game's events can be modified with new board pieces and text blocks to build up a setting or story, and allow for new features and mechanics to be implemented in the physical part of the game.

\subsection{Events}
When developing \textit{Archipelago} and by utilizing PCG in the hybrid game setting, we discovered more uses and ways to take advantage of the processing power that a digital device holds. Our main focus point in the game when it comes to PCG, was the events and the flavour of them. During developing and when reflecting on the process, there are many ways that PCG can be utilized. By having the algorithm for the events that we have, we can simply add more information into the system and be able to get new content in a way that would not be possible with only an analogue version of the game. If we add more pieces to the system, and then bits and pieces of flavour that match them, the system is able to generate completely new content, that could be seen as a traditional expansion set to the game without having to physically take up more space and time. The power of the PCG algorithm is that it allows for the continued generation of new content with minimal effort. New rewards and penalties can be added as the designer sees fit for the game, and once these rather small pieces of information are added to the XML file, everything can be implemented automatically.

The events are the place where the application delivers content to the board game which is hard and cumbersome to replicate using only analogue components. The multitude of events and the different combinations which can be achieved would require an immense set of cards in order to be comparable. 
The special events, which uses the players previous actions, are even harder to replicate. The cards would have  to have some sort of combination mechanic based on visited locations and which factions they have interacted with and how that interaction has been conducted. 

Replicating this mechanic with analogue content would quickly prove cumbersome, if not practically impossible. 
This can be seen as one of the main arguments for the use of the added digital component and a validating the use of such instead of an analogue counterpart.

\subsection{Map layout}
The layout of the map as implemented in the current state of Archipelago is what enables the game to have so many and varied events. By using PCG combined with the L-System, the game affords the players to have a vast area to explore in any direction. Although the map in its current state does not have any specific mechanics built into it, the possibilities are definitely there. As elaborated on in the discussion part, other possibilities for the map includes having quest-lines embedded within the map nodes. This was a discovery we made when playtesting the game. As it is now, the map does the job of enabling the players to move around and complete the events. But by implementing questing logic into the interpretation part of the L-System, the game can offer a much more varied play and different play styles along with encouraging more discussion and tactical play.

\subsection{Mechanics}
In regards to the map layout section above, the whole questing mechanic is one of the most interesting ones. Using event linking and maybe creating a new way for the players to explore the world is something that was found to be implementable with future work. Taking better use of the map layout as a whole seems to be in the next step, as it right now is only really a placeholder for the events. If the meta of having questlines and side-quests linked together with the map was implemented, one could use the event system and datamanager already implemented to create new stories and bring more immersion to the players. Since this was not implemented or tested, this is only speculation at this point, however, the ideas are some that definitely are worth mentioning. The concept of quests might bring thoughts about existing digital games as well as board games, but being able to experience the digital power within a physical setting is one that is rarely touched upon. With this in mind, more mechanics and concepts are easier to come up with, as there is a huge range of games, both digital and physical, that hold concepts that can be expanded upon with a hybrid combination.

%\section{Possible alternatives to our solution}
%One of the most obvious alternatives to our solution is to create a complete board game, or a complete digital game, instead of using both as a hybrid. However, this would not be in the scope of the project, as we want to explore the space between PCG and Boardgames by the use of an application and the making of a hybrid game.
%As briefly mentioned earlier, it would be nice to take more use of the map, its layout, and its possibilities. Creating quest lines and hidden events that would occur as the game progress. What would also be nice, is to have a route within the map, so that the players do not necessarily have to just go towards one node, but have one node at the start that is the first goal, then having other goals afterwards as they move through the map. 

\section{Future work}
\label{sec:future}
In this part, we will elaborate upon possible work that could be done in the future. What we would like to implement, thoughts on how to do that, and what could have been done differently in order to make this easier in the future.

\subsection{Better use of saved data}
\label{sec:savdat}
One of the big impacts this project brought to light, was the use of saved events and its data to generate new events that were based on those. In this solution, we save the events that happen over the course of the game, and we use them in a rather minimalistic way, more as a proof of concept. However, it would be interesting to see how this data could be used more efficiently over an entire game session. Generating more experience driven content, that would have a bigger impact on the players' experience, by creating other mechanics in the game, is one thought that comes to mind. 

The saved data could be used in ways which a board game could never compete. One could use the saved data to challenge the players either more or less, by knowing the number of resources they have collected, and tailoring the possibilities from that knowledge. In this way, the game would either challenge the players to use their resources wisely or give them access to a resource which has been sparse in the previous events. 

The special events could really play to what the players had been doing, in terms of where and for who, giving the players a greater feel of involvement in the world. This could be shown with some sort of faction overlay, which could change each time you dealt with the different factions. It could be possible to have some sort of background faction warfare with variables being influenced by how the players interact with the game.

Also, a story could be brewed by linking all the outcomes of the events. In the end building a short story based on the players events. Using a text algorithm of sorts to create a story based on the factions relationships and the deals made at each point of interest. 

By using the current layout, one could even imagine that the application could piece together the state of the board game, using the input from the events. Whenever the players resolve the events, they declare, indirectly the tells the application how they have used their resources. Saving this could help to give a complete look at how the players are doing and surprise the players by revealing to them in events that the application knows what they have built.

In the end, the saved data could be used in many ways to give the players a feeling of a game which reacts to their input, even without them having directly to manipulate the application with the results and state of their board.

\subsection{More options and outcomes}
At this point in time, there are a finite number of possible outcome types, rewards, and penalties. For a future version of the game, it would be interesting to see how the game would be experienced if there were more types of tokens to be played around with. It would also be nice to have more flavour text pieces that could be combined, maybe in a new way, and presented to the players. 

Right now the system is capable of combining the different parts of the game and generate events with flavour text, using a combination of the board pieces and flavour text strings.
The system is built such that more pieces could be added, along with flavour text bits for each of them, and still generate meaningful events. 

However, there is right now around 3-7 text strings for each of the board pieces, all of which can be combined which each other. This could, of course, be increased, but it would also be interesting to generate text which is capable of having some sort of interrelation between the pieces. 
This could perhaps be accomplished using some sort of grammar such as the L-systems, which could take the current text being built and use it to create a better interrelation. 
The grammar could as an example relate Alchemy points to Castle crew and base the flavour text on some sort of potions to improve the crew, or relate the alchemical potency to how it improves the machinery used in a specific event. 
Having this relation brings the game more to life and makes the players feel more immersed in the experience.

In the end, having more pieces, a variety of flavour, and ways of combining the flavour, would be like creating an expansion pack of cards that would be added to the physical board game.

\subsection{Additional features based on the map}
The current map layout generates a larger structure of floating islands which you can visit in no particular order. The map contains floating islands each with events tied to it and different locations available.

However, the map is an unused canvas which could have more features to better engage the players in moving around on the map. 
There could be included places of interest, where special quest could be implemented. A chain of quest linked in creating a short story or other sub-plots. 
As said above in section \ref{sec:savdat}, there is also a possibility of having factional warfare, using the saved events. This could both be displayed and engaged directly on the map. 
Timed events could appear based on factional changes and the events which have been resolved by the players. The events could appear on the map in a small vicinity of the player location. This would engage the players in checking out how the map situation is unfolding. 

Another way to increase the exploration is to hide the end goal. Then perhaps revealing the goal through a quest chain or through some sort of cost of materials. The general direction of the goal could be known to the player, but you have to either complete some quest or guess where the goal was through hints given throughout the game.

The chain quest could be one which spawns the goal once all the quest are completed. Having the players creating a portal to take them home after the quest is done.

What could be done to further enhance the diversity of the layouts, is to introduce some sort of genetic mutation on the grammar of the L-System. This would allow the maps to be constantly changing, and with the use of a validation, or fitness function, the maps could be tailored to better fit the different types of players.

On top of just changing the map's layout, you can add quests and puzzles into the map itself. If combining the L-System with graph grammars, the map can be made so that one island has a quest assigned to it, and that quest could, for instance, have the players move towards another side of the map, to one specific island. 
The end island could then be hidden from the players completely, and only be showed/enabled when the players had reached a certain part of the quest line.

\subsection{Other potential mechanics}
In the following subsections, other mechanics which might improve the game or bring new aspects, are described in brevity. 
\subsubsection{Faction visualization}
A feature which could be included in the future development is a visual representation of the current relation to the factions.
Currently, there is feedback from the application whenever there is a change in the relation to the factions. But this information only shows up whenever there is a change, which is somewhat problematic. 

A fix for this would be to have some access to a part of the UI which would display the current standing with the available factions. This would give the players the possibility to make a better informed decision of how to react to special events and maybe steer toward some specific relation to the factions.

\subsubsection{Event progression}
For the events, it would be interesting to see how the players would act, if there was some progression system in place, that would make it harder and harder as the game went on. And if this would be combined with perhaps some artificial intelligence in order to generate new rewards, it could be educational to see how the mechanics of the game could change along with how the players would perceive the space of possibility.

\subsubsection{Data gathering}
An option that could yield data for future development, would be to implement some sort of statistic gathering algorithms that would save data like average time of a game, moves made per game, resource and reward amounts and chances of getting these. With this information, we as the developers of the game can come up with possible solutions and changes that could improve the game and the way it is played.

\subsubsection{Other event mechanics}
For the PCG part, it could be interesting to see how the game would be perceived if the players could give the application some more specified inputs like what resources they were needing so that the event algorithm could take that into consideration and maybe generate the events with a bias towards getting more resources. Maybe even having a new mechanic that allows the players to select from different event types; i.e. fighting, researching, gathering, etc. and if they selected one of them, it would create consequences for the next events to come.

One of the big wants for future work is to see how we could make the events evolve as the players progress. Having different difficulty modes and perhaps introducing a genetic algorithm for the evolution of events and resources, rewards and penalties. This is all speculation at this point, but the mechanics it could bring with them could definitely be worth exploring.

\subsubsection{Memory based feature}
By having access to the memory and computation power that is in the digital device, calculations can be made to see which types of options the players select during an event. For instance, if one event has a third option that requires the players to have a specialization in healing, the application would then be able to calculate that the players have used at least 4 building materials, along with 2 alchemy points in order to get that room. The application would then do the calculations and save the result in the application, that the players now have 4 fewer materials, and 2 fewer alchemy points. Without the memory, this would not be possible, and the application would rather have to have set rules for how the events could be made up, or the players would have to give the application inputs so that it would know where to continue. By taking better advantage of this scenario, the application could start checking the previously gathered resources against the selected options, and calculate how well a player is doing, and by this also be able to adjust the difficulty of the game accordingly.

\subsubsection{Saving play progression}
Currently, the application does not save the players current progression and if the application is closed, the players will have to start over. Saving the current progression of the application, allowing players to load and progress from a point, instead of starting over each time. This would also ensure the session could be restored in case of power failure, so that time playing the game would not be wasted and the session lost.

\subsubsection{Previous events}
The new events could be generated based on all the previous events which have occurred. This way the application could take into account if the players were doing too badly or if a certain resource had never been shown, the application could adapt and ensure the players were not being unfairly challenged. 
This could go the other way as well, ensuring that players which have had it too easy would get more difficult events as the game progressed.