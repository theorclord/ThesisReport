\chapter{Conclusion}
The creation of \textit{Archipelago} had the purpose of exploring a new type of tabletop games using digital components to enhance their gameplay. This was done thanks to the use of PCG. This chapter will conclude the project by answering the questions at its origin. This will be done by identifying the key results of the experiment based on the constructed framework and on the knowledge gained thanks to the playtests, and the development itself. Finally, the last section will show how \textit{Archipelago} paves the way for experimentations relative to the purpose of this project.

\section{The exploration of hybrid games}
Introducing PCG to hybrid games created a much bigger perceived possibility space regarding the game's content. Compared to regular hybrid games, where there is no specific PCG involved, \textit{Archipelago}'s algorithms and structure allows for more content to be created while at the same time enabling the possibility for there to be pre-made content. 

Through the work done with developing \textit{Archipelago}, as a hybrid game, many challenges relative to the ordinary approaches to board game design were discovered. Creating such a game within an only slightly investigated field actually proved itself to be a challenge. When developing both a physical and a digital game concurrently, one of the most important factors is that the mechanics design is thought through and that both the board game and the application have equal, yet distinctly specific purposes. \textit{Archipelago} is different than already existing hybrid games such as the ones described in this paper. The first aspect underlining this difference is the fact that \textit{Archipelago} does not require inputs from the players solely for the purpose of directly feeding the application with information. This was an important mindset to keep during the development of the game in order to preserve the tabletop experience and use the digital device in a smart way, to avoid the use of such inputs.

Another principle on which \textit{Archipelago} was based on was to make the game entirely dependent on the digital device that is used to support its mechanics. This was achieved by basing an entire phase of the game on the use of the digital device and its computing powers. By creating mechanics that require both types of components and making the game collaborative, the game illustrates this approach of hybrid games without denaturing the traditional tabletop gameplay experience.

\subsection{Increased replayability and adaptability}
When developing \textit{Archipelago} and by utilizing PCG in the hybrid game setting, we discovered more uses and ways to take advantage of the processing power that a digital device holds. Our main focus point in the game when it comes to PCG was the events and the flavour associated with them. Developing and when reflecting on the process showed in what ways PCG could be useful. Having the algorithm for the events that is used by \textit{Archipelago} allows us to simply add more information into the system and be able to get new content in a way that would not be possible with only an analogue version of the game. If we add more pieces to the system, and then bits and flavour that match them, the system is able to generate a completely new content, that could be seen as a traditional expansion set to the game without having to physically take up more space and time. The power of the PCG algorithm is that it allows for the continued generation of new content with minimal effort. By using PCG combined with the L-System, the game affords the players to have a vast area to explore in any direction. New rewards and penalties can be added as the designer sees fit for the game, and once these rather small pieces of information are added to the XML file, everything can be implemented automatically. The events are the place where the application delivers content to the board game which is hard and cumbersome to replicate using only analogue components. The multitude of events and the different combinations which can be achieved would require an immense set of cards in order to be comparable. 
The special events, which uses the players previous actions, are even harder to replicate. The cards would have to have some sort of combination mechanic based on visited locations and which factions they have interacted with and how that interaction has been conducted. They also illustrate how PCG should be used to make the players feel that their actions have an influence over the game's content.

\section{Future work}
\label{sec:future}
Being an experimental concept, \textit{Archipelago} has been created as a project on which it would be possible to build upon in order to experiment various possibilities exploiting PCG. The final part of this paper will be dedicated to this purpose by listing possible implementations that are based on \textit{Archipelago}'s concept.

\subsection{Better use of saved data for player control}
\label{sec:savdat}
One of the big impacts this project brought to light, was the use of saved events and its data to generate new events that were based on those. In this solution, we save the events that happen over the course of the game, and we use them in a rather minimalistic way, more as a proof of concept. However, it would be interesting to see how this data could be used more efficiently over an entire game session. Generating more experience driven content, that would have a bigger impact on narrative and make a deeper used of the theme by creating other mechanics in the game, is one thought that comes to mind. 

The saved data could be used in ways which a board game could never compete with. One could use the saved data to challenge the players either more or less, by knowing the number of resources they have collected, and tailoring the possibilities from that knowledge. In this way, the game would either challenge the players to use their resources wisely or give them access to a resource which has been sparse in the previous events. One could even imagine using some sort of genetic algorithm to tailor the events directly to the player profiles, all built from the saved data.

The special events could really relate to what the players had been doing, in terms of where and for who, giving the players a greater feel of involvement in the world, and enhancing the feeling of relative control. This could be shown with some sort of faction overlay, which could change each time the different factions are dealt with. It could be possible to have some sort of background faction warfare with variables being influenced by how the players interact with the game. Also, a story could be brewed by linking all the outcomes of the events. In the end building a short story based on the players events. Using a text algorithm of sorts to create a story based on the factions relationships and the deals made at each point of interest. Furthermore, a progression in difficulty based on the progression of the saved data would give rise to new challenges for the players.

By using the current layout, one could even imagine that the application could piece together the state of the board game, using the input from the events. Whenever the players resolve the events, they declare, indirectly the tells the application how they have used their resources. Saving this could help to give a complete look at how the players are doing and surprise the players by revealing to them in events that the application knows what they have built.

In the end, the saved data could be used in many ways to give the players a feeling of a game which reacts to their input, even without them having directly to manipulate the application with the results and state of their board.

\subsection{More options and outcomes for diversity}
At this point in time, there are a finite number of possible outcome types, rewards, and penalties. For a future version of the game, it would be interesting to see how the game would be experienced if there were more types of tokens to be played around with. It would also be nice to have more flavour text pieces that could be combined, maybe in a new way, and presented to the players. 

Right now the system is capable of combining the different parts of the game and generate events with flavour text, using a combination of the board pieces and flavour text strings.
The system is built such that more pieces could be added, along with flavour text bits for each of them, and still generate meaningful events. 

However, there is right now around 3-7 text strings for each of the board pieces, all of which can be combined which each other. This could, of course, be increased, but it would also be interesting to generate text which is capable of having some sort of interrelation between the pieces. 
This could perhaps be accomplished using some sort of grammar such as the L-systems, which could take the current text being built and use it to create a better interrelation. 
The grammar could as an example relate Alchemy points to Castle crew and base the flavour text on some sort of potions to improve the crew, or relate the alchemical potency to how it improves the machinery used in a specific event. 
Having this relation brings the game more to life and makes the players feel more immersed in the experience.

In the end, having more pieces, a variety of flavour, and ways of combining the flavour, would be like creating an expansion pack of cards that would be added to the physical board game.

\subsection{Additional features based on the map}
Increasing the feeling of exploration is also one of the purposes that \textit{Archipelago} could explore. The current map layout generates a larger structure of floating islands which can be visited in no particular order. The map contains floating islands each with events tied to it and different locations available. However, the map is an unused canvas which could have more features to better engage the players in moving around on the map. 
There could be included places of interest, like special quests to tell stories or trading posts.
As said above in section \ref{sec:savdat}, there is also a possibility of having factional warfare, using the saved events. This could both be displayed and engaged directly on the map. Timed events could appear based on factional changes and the events which have been resolved by the players. The events could appear on the map in a small vicinity of the player location. This would engage the players in checking out how the map situation is unfolding. What could be done to further enhance the diversity of the layouts, is to introduce some sort of genetic mutation on the grammar of the L-System. This would allow the maps to be constantly changing, and with the use of a validation, or fitness function, the maps could be tailored to better fit the different types of players.


\subsection{Other potential improvements}
In the following subsections, other mechanics which might improve the game or bring new aspects, are described in brevity. 
\subsubsection{Feedback and User Experience}
Currently, the playtests have shown the importance of feedback to combine analogue and digital components as a whole. This has mainly been shown when playtesting the implementation of the faction system for the first time. The results of the final playtest clearly showed that the relevant information used to perform the event were not enough emphasized.
A fix for this would be to have some access to a part of the UI which would display the current standing with the available factions. This would give the players the possibility to make a better-informed decision of how to react to special events and maybe steer toward some specific relation to the factions.


Another improvement that has proved interesting when developing \textit{Archipelago} but fell out of the scope of its creation is the use of the digital device to display and teach the rules. This does not involve the use of PCG, but has proved to be a nice way to ameliorate the general play experience by rapidly teaching us the rules when playing \textit{XCOM: The Board Game} \cite{game:xcomtbg}. 

\subsubsection{Data collection for further improvement}
An option that could yield data for future development, would be to implement some sort of statistic gathering algorithms that would save data like average time of a game, moves made per game, resource and reward amounts and chances of getting these. With this information, we as the developers of the game can come up with possible solutions and changes that could improve the game and the way it is played.

More generally, developing \textit{Archipelago} has clearly been useful to identify the issues behind the creation of hybrid games. Both successes and failures were analysed to provide the necessary results to understand how such games must be thought and conceived as a new field to explore for designers, and new challenges to tackle for programmers by using existing methods as well as experimenting new possibilities to enhance tabletop gameplay. 
