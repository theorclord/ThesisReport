\chapter{Conclusion}
\section{The game's contribution and the pcg involvement. }

The possibilities that PCG provide during a project like this are immense. But building the program and discovering all the features wanted is a great task. In \textit{Archipelago} PCG was the biggest source of giving the players a setting in which they would immerse themselves. Since the events was the main connection between the board game and the application, it was only natural that this was the area that would bring the game to a whole. Having the events being generated procedurally made it possible to create a setup for future expansion of the game in terms of easily being able to add more smaller parts of content into the system. By having this fragmented structure, we were able to make the game system in a way that would allow for different themes and stories to be built. The game's events can be modified with new board pieces and text blocks to build up a setting or story, and allow for new features and mechanics to be implemented in the physical part of the game.\\\\

- What did PCG enable in terms of:
	- Story
	- Events
	- Mechanics
	- Expansions
	- "Card" generation
	- Map Layouts
	
-Events:
When developing \textit{Archipelago} and by utilizing PCG in the hybrid game setting, we discovered more uses and ways to take advantage of the processing power that a digital device holds. Our main focus point in the game when it comes to PCG, was the events and the flavour of them. During developing and when reflecting on the process, there are many ways that PCG can be utilized. By having the algorithm for the events that we have, we can simply add more information into the system and be able to get new content in a way that would not be possible with only an analogue version of the game. If we add more pieces to the system, and then bits and pieces of flavour that match them, the system is able to generate completely new content, that could be seen as a traditional expansion set, to the game without having to physically take up more space and time. The power of the PCG algorithm is that it allows for continued generation of new content with minimal effort. New rewards and penalties can be added as the designer sees fit for the game, and once these rather small pieces of information is added to the XML file, everything can be implemented automatic. 



\section{Relate work to initial questions posed in the introduction}

