\documentclass[a4paper,11pt]{article}
\usepackage[utf8]{inputenc}
\usepackage{graphicx}
\usepackage[english]{babel}
\usepackage[vmargin=3.5cm, top=2cm]{geometry}
\usepackage[linktocpage=true]{hyperref}
\usepackage{enumitem}
\usepackage{longtable}
\usepackage{pdfpages}
\usepackage{float}
\usepackage{hyperref}
\usepackage[section]{placeins}
\usepackage{listings}
\hypersetup{
    colorlinks,
    citecolor=black,
    filecolor=black,
    linkcolor=black,
    urlcolor=black
}
\newcommand{\tmtable}{\begin{longtable}{ p{2.7cm} p{10cm} }}
\newcommand{\tmtableend}{\end{longtable}}

\begin{document}
\lstset{language=C}  
\begin{titlepage}

\centering \parindent=0pt
\newcommand{\HRule}{\rule{\textwidth}{1mm}}
\vspace*{\stretch{1}} \HRule\\[1cm]\large\bfseries
Thesis title\\[0.7cm]
\large Masters Project\\[1cm]
\HRule\\[1cm]

\begin{figure}[h]
	\centering
    \includegraphics[width=1\textwidth]{Images/FrontPage.png}
    \label{fig:frontPage}
\end{figure}
\large by 
\\Mikkel Stolborg (msto@itu.dk)
\vspace*{\stretch{2}} \normalsize
\begin{flushleft}
IT University of Copenhagen \\
Supervisor\\
Julian Togelius\\
\today \end{flushleft}
\end{titlepage}

\begin{abstract}
Some intro about what procedural content generation is.
Some intro about what board games is.
Intro to our project.
What have been produced.
Conclusion.
\end{abstract}
\pagebreak
\tableofcontents
\pagebreak
\section{Introduction}
Introduction to the ideas behind the project. 

Thesis layout.
\pagebreak
\section{Background}
Background for our project. 
Related work with PCG.
Algorithms used.
Board games of a similar manner.

\section{Methods}
How we created our project. 
The algorithms used and the tools used.
The design process.

\section{Discussion}
Here we will discuss the project result.

\section{Conclusion}
Here we will conclude on the project.

\pagebreak
\bibliography{sources}{}
\bibliographystyle{plain}

\pagebreak


\end{document}
